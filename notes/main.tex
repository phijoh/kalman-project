\documentclass[american]{scrartcl}

    \newcommand{\lang}{en}

    \usepackage{babel}
    \usepackage[utf8]{inputenc}

    \usepackage{csquotes}

    \usepackage{amsmath, amssymb, mathtools}
    \usepackage{bm}

    \usepackage{graphicx}
    \usepackage{tikz} 

    \usepackage{subfiles} % Load last

    % Formatting
    \setlength{\parindent}{0em}
    \setlength{\parskip}{0.5em}
    \setlength{\fboxsep}{1em}

    % Graphs
    \usetikzlibrary{positioning, arrows.meta, calc, decorations.markings, math, matrix}

    \tikzset{
        main 
        node/.style={
            circle, draw,minimum size=2cm,inner sep=3pt
        },
    }
    
    % Math commands
    \newcommand{\E}{\mathbb{E}}
    \newcommand{\R}{\mathbb{R}}
    \newcommand{\V}{\mathbb{V}}

    \newcommand{\matr}[1]{\bm{#1}}
	\newcommand{\set}[1]{\left\{#1\right\}}
	\newcommand{\diag}{\text{diag}}

    \DeclareMathOperator{\Tr}{Tr}

    \DeclarePairedDelimiter\abs{\lvert}{\rvert}%
    \DeclarePairedDelimiter\norm{\lVert}{\rVert}%

    \usepackage[
        bibencoding=utf8, 
        style=apa
    ]{biblatex}
    
    \title{Notes}

    \author{Philippa Johnson and Andrea Titton}
    
\begin{document}

\maketitle

\section{Ratio estimation}

If the estimation is done for $\rho :=  \frac{\sigma^2_p}{D_v}$ then,

\begin{equation}
    \gamma = \left( 1 + \rho \right)^{-1}
\end{equation}

and,

\begin{equation}
    \begin{split}
        \V(v_u)^{-1} = \frac{1}{\sigma^2_p} + \frac{1}{D_v} &= \frac{\sigma^2_p + D_v}{\sigma^2_p \cdot D_v} \\
        &= \frac{1 + \rho}{\sigma^2_p}
    \end{split}
\end{equation}


\section{Damping}

\begin{equation}
    \begin{split}
        \gamma &= \left(1 + \frac{D_v}{\sigma^2_p}\right)^{-1} \\
        \gamma^{-1} - 1 &= \frac{D_v}{\sigma^2_p} \\
        \sigma^2_p &= \frac{D_v}{\gamma^{-1} - 1}
    \end{split}
\end{equation}

For example, $\gamma = 0.89$,

\begin{equation}
    \sigma^2_p \approx \frac{D_v}{0.124}
\end{equation}

\end{document}